{\documentclass [10pt,a4paper]{book}

\begin{document}

\begin{flushright}
NET-BASED CONSENSUS TECHNIQUES  \textbf{129}
\end{flushright} 
from relatively homogeneous groups rarely results in new information (Collins, Osborne. Ratcliffe, Millar,  Duschl, 2001). Generally, the panel of experts should be representative of diverse backgrounds with respect to experience and expertise. The heterogeneity of the participants must be preserved to assure validity of the results. That is, the possibility of domination by one or several experts, causing a bandwagon effect, must be removed. This can be achieved through ensuring that the end-of-round summaries include all voices. In addition, the researcher's views and partialities to cer-tain problems should never be imposed on the group respondents—especially at the expense of other perspectives in relation to the problem.


The number who agree to participate may be considerably lower than the num-ber who are approached—Delphi studies are notorious for requiring considerable time commitments. To keep those that agree to participate from dropping out, the researcher must attempt to walk a fine line between allowing a wide latitude in the con-tribution of information and summarizing each round. Specifically, the researcher should conduct the research in a way that will not result in an overwhelming amount of information that will require an excessive amount of time on the part of the partic-ipants when reviewing the summaries. Flow to limit the summaries for an efficient communication process without sacrificing the participant's contribution is an issue that brings ethics, validity, and trustworthiness into play. The researcher must use his or her personal discretion to decide what will be included in the summaries. Personal discretion can be translated to personal preference or personal bias. To be ethical and maintain validity, the researcher must state whose and what parts of the contributions were limited as well as why. To promote trustworthiness, Glesne and Peshkin (1992) suggest that the researcher enlist an outsider to audit the summaries. 


Finally, the reliability is also related to the size of the group. That is, the relia-bility of the group responses increases with the size of the group (Dalkey, 1972). Yet, one should not be too concerned, in a statistical sense, about the group size because inferential statistics are not used (Dalkey in Shearin, 1995). That is, the objective in selecting persons to serve on a panel is to choose or find people having special knowl-edge or expertise in the research area. Allen (in Shearin, 1995) claims, for the reasons cited by Dalkey, a definitive sample size is not required for an effective study, but also states in another publication (Allen, 1971)) that an ideal size is a panel of thirty people. This is also in contrast with Martino (1972) who writes that a panel of fifteen, consist-ing of a cross section of experts from a particular field, is sufficient for reliable results. According to Martino, if the researcher has fifteen or more responses to the question-naire, the study can be considered to adequately meet the question of reliability. Sim-ilarly, according to Tersine and Riggs (1990), a panel of ten to fifteen members has been judged sufficient for producing effective results.
\begin{flushleft} 
\textbf{Formatting the Questions}
\end{flushleft} 
Issues relating to formatting a Web or mail survey are covered in Chapter I I and apply equally to the surveys used in an online Delphi. Figure 9.1 is an example of a Delphi question that was developed and used by Kathryn A. Kennedy from the University of British Columbia fora Ph.D. (2002) study related to rate of adoption of online courses. 
\end{document}	