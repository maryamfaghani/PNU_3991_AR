\documentclass [12pt]{beamer}
\usepackage{xcolor}	
\usepackage{tikz}
\usetheme{Warsaw}
\useoutertheme{infolines}
\usepackage{ragged2e}
\begin{document}

\section*{kholase safahat 127...129}
\subsection*{maryam faghani}	
\begin{frame}
\justifying	
Wikis is not used as a tool in a formal e-research project - but it could be!  Developing a network-based research project focused on net-based manufacturing is the first step in any research project planning operational steps.  
\end{frame}	
\begin{frame}
\justifying	
For a consensus study the plan revolves around setting research questions, selecting a sample of participants, deciding on participants' interaction methods, deciding on analysis techniques, and deciding how to publish results.
If the chosen technique allows participants to know and interact directly with other members (as often happens in nominal group online processes).
\end{frame}
\begin{frame}
\justifying	
large differences in status may lead to an honest and open discussion.  Disrupt.
The possibility of mastery by one or more specialists, which causes the effect of the wagon band, should be eliminated.  This can be achieved by ensuring that round end summaries include all sounds.
\end{frame}
\end{document}		