{\documentclass [10pt,a4paper]{book}

\begin{document}

\begin{flushright}
	
NET-BAED CONSENSUS TECHNOUES      \textbf{127}
	
\end{flushright}			
exclusively on the Internet, though certain portions of face-to-face consensus confer-
ences have been webc-ast (see hrtp1//www.nih.govinews/prioct2000/nmar-25.hun), and 
there are certain similarities between consensus conferences and virtual conferences.

\begin{flushleft}
\textbf{WIKI SYSIEM}
\end{flushleft}
A final emerging form of consensus-building technique that is native to the Net is the WIKI Web system designed by Ward Cunningham (http://www.c2.com/cgi/ wiki?WelcomeVisitors). A WIKI site allows selected, or all, users to jointly (and anonymously if they choose) edit hypertext markup language (I ITML) documents. This means that any user can edit, delete, add, or otherwise alter the page on which the group is working. This may seem like a formula for anarchy (like the Web?), but it cap-tures some of the original freewheeling spirit of the Net and can support a unique form of reflective, anonymous, and open-ended discourse. Peter Mercel, an early WIKI developer and contributor, described the ambiance of a WIKI group as "insecure, indiscriminate, user-hostile, slow, and full of difficult, nit-picking people. Any other online community would count each of these strengths as a terrible flaw. Perhaps WIKI works because the other online communities don't" (undated WIKI page at http://www.c2.comicgi/wiki?WhyWikiWorks). 


The major difference between a Net-based discussion group and a WIKI is the communal creation of HTML documents. Thus, the output of the group discourse is a series of continuously changing documents, rather than a long discussion or a quan-tified set of consensual agreements. Consensus has been reached when all participants are satisfied with the text and the format of the group document. 


The original WIKI software maintains some of the critical components of earlier consensus data-gathering techniques, but it is much more informal and spontaneous and lacks means for quantitatively calculating differences amongst panel members. WI1Us have been adopted by a variety of organizations and the process has evolved with the development of various sets of rules (such as restricted membership) for dif-ferent purposes. For example, an interesting site related to architecture can be found at the Three Dots site at http://www.threedots.org/. Three Dots allows for voting and in other ways attempts to maintain a heritage that evolved from Delphi Methods. To our knowledge WIKIs have not been used as a tool in a formal e-research project—but they could be! 


\begin{flushleft}
\textbf{DEVELOPING A NET-BASED CONSENSUS-BUTIDING RESEARCH PROJECT}
\end{flushleft}
The first step in any research decign is to plan the operational steps. For a consensus study, this plan revolves around setting the research question(s), selecting a participant sample, deciding on the method(s) of participant interaction, deciding on the analysis techniques, and deciding how to disseminate results. We discuss each of these components in turm.

\end{document}